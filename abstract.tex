\chapter*{ABSTRACT}
The Züs cryptocurrency platform, built on decentralized storage technology, faces challenges in ensuring the security of user assets and providing a seamless user experience. Key management, particularly in the context of cryptocurrency wallets, poses significant risks, as compromised private keys can lead to the loss of funds. Additionally, the lack of user-friendly authentication mechanisms can hinder widespread adoption. \\ \\
This project focuses on enhancing the security and usability of the Züs platform by implementing a split key wallet protocol and developing a dedicated mobile authenticator. The split key wallet protocol improves key management and reduces the risk of private key compromise by splitting the key into multiple components stored on separate devices. The mobile authenticator provides an additional layer of security for user authentication and transaction signing. \\ \\
The implementation leverages the BLS signature scheme, which supports key splitting and aggregation, offering improved efficiency compared to other signature schemes like Schnorr and ECDSA. The system architecture consists of the Züs platform, split key wallet, mobile authenticator, and cryptographic library components. \\ \\
The split key wallet protocol involves key generation, key splitting, component distribution, and transaction signing steps. The mobile authenticator, developed using the native Android & iOS framework, provides secure storage, biometric authentication, transaction signing, and push notification capabilities. \\ \\
The project demonstrates the feasibility and effectiveness of the split key wallet protocol and mobile authenticator in enhancing the security and usability of the Züs platform. The implementation details, performance analysis, and comparative evaluation provide valuable insights for researchers and developers working on similar projects in the blockchain and cryptocurrency domain. \\ \\
The contributions of this project include enhancing the security of private key management, developing a user-friendly mobile authenticator, demonstrating the efficiency of the BLS signature scheme, and providing a comprehensive system architecture and implementation details. \\ \\
Future work can explore extending the split key wallet protocol to support multiple devices, integrating secure multi-party computation techniques, conducting user studies, investigating scalability and performance, and adapting the solution to other blockchain and cryptocurrency platforms. \\  \\
Overall, this project successfully addresses the challenges of key management and user authentication in the Züs cryptocurrency platform, contributing to the advancement of secure and user-friendly solutions in the field of decentralized storage and cryptocurrency ecosystems.
\cleardoublepage
