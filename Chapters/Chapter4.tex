% Chapter 4
\chapter{Implementation}
\label{Chapter4}
\section{Technologies Used}
The implementation of the split key wallet protocol and mobile authenticator for the Züs platform involves the following technologies:
\begin{itemize}
    \item Programming Languages: The core components of the system are implemented using Golang and Solidity programming languages.
    \item Cryptographic Libraries: The BLS signature scheme is implemented using the Apache Milagro Cryptographic Library (AMCL), which provides efficient implementations of pairing-based cryptography.
    \item Mobile Development Framework: The mobile authenticator is developed using the native Android, iOS and electron framework, enabling support for iOS, Android and Desktop devices.
    \item Secure Storage: The key components and sensitive data are stored securely using hardware-backed keystores, such as the Secure Enclave on iOS and the Keystore on Android.
\end{itemize}
\section{Smart Contract Development}
The Züs platform utilizes smart contracts to facilitate decentralized storage and cryptocurrency transactions. The split key wallet protocol is integrated into the existing smart contract infrastructure. The main modifications include:
\begin{itemize}
    \item Signature Verification: The smart contracts are updated to support the verification of BLS signatures, ensuring the integrity of transactions.
    \item Key Management: The smart contracts handle the registration and management of public keys associated with the split key wallets.
\end{itemize}
\section{Wallet Integration}
The split key wallet protocol is integrated into the existing Züs wallet implementation. The main changes include:
\begin{itemize}
    \item Key Generation: The wallet is modified to generate BLS key pairs and split the private key into multiple components.
    \item Partial Signature Generation: The wallet is updated to generate partial signatures using the key component stored on the primary device.
    \item Communication with Mobile Authenticator: The wallet establishes a secure communication channel with the mobile authenticator for transmitting partial signatures and receiving the final signature.
\end{itemize}
\section{Mobile Authenticator Development}
The mobile authenticator is developed as a standalone mobile application using the Native frameworks for each platform. The main components of the mobile authenticator include:
\begin{itemize}
    \item User Interface: The user interface is designed to provide a seamless and intuitive experience for authentication and transaction approval.
    \item Biometric Authentication: The mobile authenticator integrates biometric authentication (e.g., fingerprint or facial recognition) to ensure the security of user interactions.
    \item Secure Storage: The key component and other sensitive data are stored securely using the hardware-backed keystore available on the mobile device.
    \item Push Notifications: The mobile authenticator integrates with the Züs platform's notification system to receive real-time alerts for pending transaction approvals.
\end{itemize}
\section{Integration Testing}
Rigorous integration testing is performed to ensure the smooth functioning of the split key wallet protocol and mobile authenticator within the Züs platform. The testing scenarios include:
\begin{itemize}
    \item Key Generation and Splitting: Verifying the correctness of key generation and the splitting of the private key into multiple components.
    \item Transaction Signing: Testing the end-to-end process of initiating a transaction, generating partial signatures, and combining them to produce the final signature.
    \item Mobile Authenticator Functionality: Validating the mobile authenticator's user interface, biometric authentication, secure storage, and push notification capabilities.
    \item Error Handling and Recovery: Testing various error scenarios, such as network disruptions or device failures, and ensuring proper error handling and recovery mechanisms are in place.
\end{itemize}
The implementation phase involves close collaboration between the development team and the Züs platform stakeholders to ensure seamless integration and adherence to the platform's security and performance requirements.
