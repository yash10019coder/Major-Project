% Chapter 3
\chapter{Methodology}
\label{Chapter3} % For referencing the chapter elsewhere, use \ref{Chapter3}
\section{System Architecture}
The proposed system architecture for the split key wallet protocol and mobile authenticator integration with the Züs platform consists of the following components:
\begin{itemize}
\item Züs Platform: The core platform that provides decentralized storage and cryptocurrency functionalities.
\item Split Key Wallet: A wallet implementation that splits the private key into multiple components and stores them on separate devices.
\item Mobile Authenticator: A dedicated mobile application that acts as an additional authentication factor and facilitates transaction signing.
\item Cryptographic Library: A library that provides the necessary cryptographic primitives, including the BLS signature scheme.
\end{itemize}
\section{Split Key Wallet Protocol}
The split key wallet protocol is designed to enhance the security of private key management by splitting the private key into multiple components. The protocol consists of the following steps:
\begin{enumerate}
\item Key Generation: The user generates a private key and corresponding public key using the BLS signature scheme.
\item Key Splitting: The private key is split into two components using a threshold secret sharing scheme, such as Shamir's secret sharing.
\item Component Distribution: One component is stored on the user's primary device (e.g., laptop), while the other component is securely transferred to the mobile authenticator.
\item Transaction Signing: To sign a transaction, the user initiates the signing process on the primary device, which generates a partial signature using its key component. The partial signature is then securely transmitted to the mobile authenticator, which combines it with its key component to produce the final signature.
\end{enumerate}
\section{Mobile Authenticator}
The mobile authenticator is a dedicated mobile application that serves as an additional authentication factor and facilitates transaction signing. The main features of the mobile authenticator include:
\begin{itemize}
\item Secure Storage: The mobile authenticator securely stores the user's key component and any other sensitive information.
\item Authentication: The mobile authenticator provides an additional layer of authentication, requiring the user to confirm their identity through biometric authentication or a PIN.
\item Transaction Signing: The mobile authenticator receives partial signatures from the primary device and combines them with its key component to generate the final signature for transaction approval.
\item Notifications: The mobile authenticator sends real-time notifications to the user for pending transaction approvals and other important events.
\end{itemize}
\section{BLS Signature Scheme}
The BLS signature scheme is chosen for the implementation due to its support for key splitting and aggregation. The BLS scheme offers the following advantages:
\begin{itemize}
\item Short Signatures: BLS signatures are compact, reducing the storage and transmission overhead.
\item Aggregation: Multiple BLS signatures can be aggregated into a single signature, enabling efficient verification of multiple transactions.
\item Key Splitting: BLS private keys can be split into multiple components, facilitating the implementation of the split key wallet protocol.
\end{itemize}
\section{Security Considerations}
The security of the proposed system relies on several assumptions and best practices:
\begin{itemize}
\item Secure Key Storage: The key components must be stored securely on the respective devices, protecting against unauthorized access.
\item Secure Communication: The communication channels between the primary device and the mobile authenticator must be encrypted and authenticated to prevent eavesdropping and tampering.
\item Device Security: Users are responsible for maintaining the security of their devices, including regular software updates and protection against malware.
\item Backup and Recovery: Mechanisms for secure backup and recovery of key components must be in place to prevent permanent loss of access to funds.
\end{itemize}
