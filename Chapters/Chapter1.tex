% Chapter 1
\chapter{Introduction}
\label{Chapter1}
In recent years, the adoption of decentralized storage and cryptocurrency platforms has grown significantly. However, ensuring the security of user assets and providing a seamless user experience remain critical challenges. Key management, especially in the context of cryptocurrency wallets, poses significant risks, as compromised private keys can lead to the loss of funds. Additionally, the lack of user-friendly authentication mechanisms can hinder the widespread adoption of these platforms.
Züs is a decentralized cloud storage platform that aims to provide secure, reliable, and scalable storage solutions while leveraging blockchain technology. The platform introduces a new decentralized finance (DeFi) model based on cloud storage, allowing users to earn steady income by staking their tokens and participating in the storage ecosystem.
This project focuses on enhancing the security and usability of the Züs cryptocurrency platform by implementing a split key wallet protocol and developing a dedicated mobile authenticator. The split key wallet protocol improves key management and reduces the risk of private key compromise by splitting the key into multiple components stored on separate devices. The mobile authenticator provides an additional layer of security for user authentication and transaction signing.
The main objectives of this project are as follows:
\begin{enumerate}
\item Implement a split key wallet protocol to enhance the security of private key management in the Züs platform.
\item Develop a dedicated mobile authenticator to provide an additional layer of security for user authentication and transaction signing.
\item Evaluate the performance and usability of the implemented solution and compare it with existing approaches.
\end{enumerate}
The scope of this project is limited to the implementation of the split key wallet protocol and mobile authenticator specifically for the Züs platform. The solution may not be directly applicable to other blockchain or cryptocurrency platforms without necessary modifications.
The limitations of the project include the reliance on the security of the underlying cryptographic primitives and the assumption that users will follow best practices in securing their devices and private key components.
The rest of the report is organized as follows:
\begin{itemize}
\item Chapter 2 presents a literature review of related work in the areas of key management, authentication, and signature schemes in blockchain and cryptocurrency systems.
\item Chapter 3 describes the methodology adopted for the implementation of the split key wallet protocol and mobile authenticator, including the system architecture and design choices.
\item Chapter 4 discusses the implementation details, including the technologies used, code snippets, and integration with the Züs platform.
\item Chapter 5 presents the results and analysis of the implemented solution, including performance evaluation and comparison with existing approaches.
\item Chapter 6 concludes the report, summarizing the key findings and contributions, and outlining potential future work.
\end{itemize}
