\chapter{Literature Review}

\section{Key Management in Blockchain and Cryptocurrency Systems}
Key management is a critical aspect of blockchain and cryptocurrency systems, as the security of user assets heavily relies on the integrity of private keys. Various approaches have been proposed to enhance the security of key management, including hierarchical deterministic wallets [1], multi-signature wallets [2], and threshold signature schemes [3].

\section{Authentication Mechanisms in Decentralized Systems}
Authentication in decentralized systems poses unique challenges due to the absence of a central authority. Several authentication mechanisms have been explored, such as decentralized identity systems [4], biometric authentication [5], and hardware-based authentication [6].

\section{Signature Schemes for Blockchain and Cryptocurrency}
Signature schemes play a vital role in ensuring the integrity and non-repudiation of transactions in blockchain and cryptocurrency systems. The most commonly used signature schemes include the Elliptic Curve Digital Signature Algorithm (ECDSA) [7] and the Schnorr signature scheme [8]. Recently, more advanced signature schemes, such as the BLS signature scheme [9] and the Boneh-Lynn-Shacham (BLS) signature scheme [10], have gained attention due to their support for key aggregation and efficient verification.

\section{Split Key Protocols and Threshold Cryptography}
Split key protocols and threshold cryptography have been studied extensively in the context of secure key management and distributed trust. Shamir's secret sharing [11] and Blakley's secret sharing [12] are well-known techniques for splitting a secret into multiple shares, requiring a threshold number of shares to reconstruct the original secret. These techniques have been applied to various cryptographic primitives, including encryption [13] and digital signatures [14].

\section{Mobile Authenticators and Two-Factor Authentication}
Mobile authenticators and two-factor authentication (2FA) have become increasingly popular for enhancing the security of user authentication. Various schemes have been proposed, such as time-based one-time passwords (TOTP) [15], push notifications [16], and QR code-based authentication [17]. These schemes provide an additional layer of security by requiring users to possess a secondary device or token in addition to their primary authentication factor (e.g., password).

