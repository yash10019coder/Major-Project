% Chapter 5
\chapter{Conclusion and Future Work} % Main chapter title
\label{Chapter5} % For referencing use \ref{Chapter5}
\section{Summary}
This project aimed to enhance the security and usability of the Züs cryptocurrency platform by implementing a split key wallet protocol and developing a dedicated mobile authenticator. The split key wallet protocol improves key management and reduces the risk of private key compromise by splitting the key into multiple components stored on separate devices. The mobile authenticator provides an additional layer of security for user authentication and transaction signing.
The implementation leverages the BLS signature scheme, which supports key splitting and aggregation, offering improved efficiency compared to other signature schemes like Schnorr and ECDSA. The system architecture consists of the Züs platform, split key wallet, mobile authenticator, and cryptographic library components.
The split key wallet protocol involves key generation, key splitting, component distribution, and transaction signing steps. The mobile authenticator, developed using the Flutter framework, provides secure storage, biometric authentication, transaction signing, and push notification capabilities.
The implementation phase involved smart contract development, wallet integration, mobile authenticator development, and rigorous integration testing. The technologies used include Rust, Solidity, Apache Milagro Cryptographic Library, and hardware-backed keystores for secure storage.
\section{Contributions}
The main contributions of this project are as follows:
\begin{itemize}
\item Enhancing the security of private key management in the Züs platform through the implementation of the split key wallet protocol.
\item Developing a user-friendly mobile authenticator that provides an additional layer of security for user authentication and transaction signing.
\item Demonstrating the feasibility and efficiency of the BLS signature scheme for key splitting and aggregation in a real-world cryptocurrency platform.
\item Providing a comprehensive system architecture and implementation details that can serve as a reference for similar projects in the blockchain and cryptocurrency domain.
\end{itemize}
\section{Future Work}
While this project successfully achieves its objectives, there are several areas for future work and improvement:
\begin{itemize}
\item Extending the split key wallet protocol to support multiple devices and a higher threshold of key components for increased security.
\item Exploring the integration of secure multi-party computation techniques to enhance the privacy and security of the key splitting and transaction signing processes.
\item Conducting extensive user studies to gather feedback and improve the usability and user experience of the mobile authenticator.
\item Investigating the scalability and performance of the split key wallet protocol and mobile authenticator in large-scale deployments with a high volume of transactions.
\item Adapting the split key wallet protocol and mobile authenticator to other blockchain and cryptocurrency platforms to promote wider adoption and interoperability.
\end{itemize}
\section{Conclusion}
In conclusion, this project successfully implements a split key wallet protocol and mobile authenticator for the Züs cryptocurrency platform, enhancing the security and usability of key management and user authentication. The implementation demonstrates the effectiveness of the BLS signature scheme for key splitting and aggregation, and provides a solid foundation for further research and development in the field of blockchain and cryptocurrency security.
The project contributes to the growing body of knowledge in decentralized storage and cryptocurrency ecosystems, and offers practical insights for developers and researchers working on similar projects. With the increasing adoption of blockchain technology and the need for secure and user-friendly solutions, the concepts and techniques explored in this project have the potential to drive innovation and shape the future of decentralized systems.
